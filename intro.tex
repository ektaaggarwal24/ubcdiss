%% The following is a directive for TeXShop to indicate the main file
%%!TEX root = diss.tex

\chapter{Introduction}
\label{ch:Introduction}

\section{Motivation}
\label{sec:Motivation}
\ac{CPS} are systems consisting of software and physical components knitted together and which are also tightly coupled to the environment in which operate. \ac{CPS} operate by taking in parameters from the physical world and software commands, and producing actions which alter the state of the environment around them. Some examples of \ac{CPS} are smart grids, smart automobiles and drones. Lately, CPS have gained ubiquitous popularity among the masses, which has led to abrupt increase in the usage of these devices in our day-to-day life. These \ac{CPS} are different from the typical computer systems because they operate in a physical environment and their properties must conform to laws of physics. Also, \ac{CPS} are built to solve a specific problem and are not multi-purpose like the typical computer systems.

As \ac{CPS} are built keeping in mind a particular output, they are often lacking in computing aspects such as memory, computational power and battery. Therefore, due to these constraints and the desire to produce a large number of \ac{CPS}, the security aspect is lagging behind.[ADD REF] As a result CPS have been targets of security attacks due to their safety-critical nature and relative lack of protection. The advent of interconnected \ac{CPS} to the Internet (also known as the Internet of Things) has exacerbated their vulnerability as they obviate the need for attackers to have physical access to the \ac{CPS}. Attacks on \ac{CPS} such as the smart grid~\cite{skopik2012survey}, smart cars~\cite{checkoway2011comprehensive} and smart medical devices~\cite{leavitt2010researchers,radcliffe2011hacking} have been demonstrated in the recent past. 
Therefore, there is a compelling need to protect \ac{CPS} from security attacks. 

\section{Approach}
\label{sec:Approach}

\ac{IDS} have been used for protecting computer systems from security attacks, including CPS~\cite{lu2015towards, mitchell2015behavior, bernieri2016testbed}. \ac{IDS} work by monitoring the activity of the system for which they are deployed and raise alarm when they detect a malicious intent. Traditional forms of \ac{IDS} are signature-based, where signatures of known attacks are compared against the operations of the system to identify attacks. Unfortunately, signature-based \ac{IDS} are a poor fit for \ac{CPS} as the attacks are often tailored to each kind of \ac{CPS}, and hence cannot be described by generic signatures. Further, due to the remote and often disconnected nature of their operation, the attack database in \ac{CPS} cannot be updated frequently unlike traditional computer systems. Finally, a motivated attacker can launch hitherto unknown attacks against a \ac{CPS}, thereby evading detection by signature-based schemes. 

In contrast to signature-based \ac{IDS}, anomaly based \ac{IDS} extract a model of a system's behavior and detect any deviations from the extracted model as an attack. Such \ac{IDS} do not need an attack database, and can hence detect hitherto unknown attacks. Because \ac{CPS} have constrained behaviors, it is often straightforward to derive anomaly based \ac{IDS} for them, making these systems a good match for \ac{CPS}. Unfortunately, anomaly based systems exhibit high rates of false-positives in practice, as learning a stable model of the system is often challenging. Therefore, some researchers have proposed using physics-based models for anomaly detection models for intrusion detection in \ac{CPS}~\cite{ mitchell2012specification,mitchell2014adaptive,choudhari2013stability,chen2018learning,zohrevand2016hidden}. The notion is that because \ac{CPS} interact closely with their physical environments, they need to follow laws of physics, which can in turn be used as the detection model. 
Efforts have been made to use the physical properties of the power grid~\cite{choudhari2013stability,paul2014unified}, \ac{UAV}~\cite{mitchell2012specification} and water treatment systems~\cite{adepu2016using} to build a model which represents the expected behavior of the \ac{CPS}. However, in the prior work, the \ac{IDS} is designed specifically for a particular \ac{CPS}. Therefore, the above solutions cannot be easily generalized to other \ac{CPS}, as the process of finding an appropriate model is both time consuming and effort intensive for developers. 

In this work, the {\em logical correlations} among the physical properties of the \ac{CPS} is considered as the model for anomaly-based \ac{IDS}. The hypothesis is that physical properties exhibit deterministic and predictable correlations among themselves, as they have to adhere to the laws of physics. For example, consider the case of an \ac{UAV}, which needs to follow Newton's laws of motion during flight. Some physical properties of an \ac{UAV} are: distance traveled, altitude, speed, battery life left and flight time. When an \ac{UAV} is flying at a fixed altitude, it has a non-zero speed due to which the distance traveled and flight time increases, while the battery life left in the \ac{UAV} decreases. These relationships encompass the logical correlations among the physical properties of the \ac{UAV}. If during flight it is observed that the battery life left in the \ac{UAV} is not decreasing while the speed of the \ac{UAV} is non-zero, this would imply that there is some anomaly in the system, which potentially indicates an attack.

\section{Contributions}
\label{sec:Contributions}

This research proposes a generic intrusion detection system capable of detecting security attacks by inferring the logical correlations of the system and checking if they adhere to a predefined framework. \ac{HMM}  are used to automatically infer the logical correlations among the physical properties of the system with no a priori knowledge of the physical laws adhered to by the system or any intervention by the programmer. The \ac{HMM} identifies a state as malicious by detecting either an undesired data correlation or lack of an expected data correlation among its physical properties. \ac{HMM} are used as they are good at detecting outliers, and are typically used to model time-based systems(~\autoref{sec:HMM_explain}).  
Though other papers have used logical correlations to detect anomalies~\cite{iturbe2017feasibility,krotofil2015process,chen2018learning,zohrevand2016hidden}, none of them have used \ac{HMM} as the core to build a generic \ac{IDS}. \textit{To the best of my knowledge, \ac{CORGIDS} is the first generic intrusion detection system which uses \ac{HMM} to infer logical correlations exhibited by the system to determine if an intrusion has occurred.}


The contributions are:
\begin{enumerate}
\item Proposition of the use of logical correlations among the physical properties of a \ac{CPS} to detect intrusions, and use \ac{HMM} to infer the logical correlations.
\item Designed a \acf{CORGIDS}, using \ac{HMM} to detect intrusions. Also, demonstrated its use on two \ac{CPS} test-beds, namely i) an \ac{UAV}, and ii) a \acf{SAP} platform.
\item Evaluated the effectiveness of \ac{CORGIDS} by performing five targeted attacks on the above mentioned \ac{CPS}. Found that \ac{CORGIDS} is successfully able to detect all five attacks, and has lower \acf{FP} and \acf{FN} rates than other intrusion detection techniques.
\end{enumerate}
%%%%%%%%%%%%%%%%%%%%%%%%%%%%%%%%%%%%%%%%%%%%%%%%%%%%%%%%%%%%%%%%%%%%%%
%%%%%%%%%%%%%%%%%%%%%%%%%%%%%%%%%%%%%%%%%%%%%%%%%%%%%%%%%%%%%%%%%%%%%%
%%%%%%%%%%%%%%%%%%%%%%%%%%%%%%%%%%%%%%%%%%%%%%%%%%%%%%%%%%%%%%%%%%%%%%

\endinput
