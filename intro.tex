%% The following is a directive for TeXShop to indicate the main file
%%!TEX root = diss.tex

\chapter{Introduction}
\label{ch:Introduction}

\section{Motivation}
\label{sec:Motivation}

\ac{CPS} are embedded systems consisting of software and physical components which collaborate and are tightly coupled to the environment in which operate. \ac{CPS} operate in a closed loop fashion which involves sensing of the current environmental conditions by the sensors, these readings are then passed as input to the controller which then based on the control logic sends actuation commands to the actuators in the \ac{CPS}. After the action has been taken, the sensors again take readings and this loop continues infinitely until the system is functioning. The rapid growth of \ac{CPS} has led to an abrupt increase in the usage of these devices in our day-to-day life. \ac{CPS} such as smart meters are used in smart grids\cite{karnouskos2011cyber, ericsson2010cyber} for recording and digitally sending the meter readings to the energy supplier for more reliability and ease of data collection. Autonomous cars\cite{checkoway2011comprehensive, yang2014vehicle} are also gaining popularity as major car manufacturers add the option of driving the car in an autonomous mode, with little or no input from humans. \ac{UAV}\cite{javaid2012cyber, mohammed2014uavs} embedded with camera, global positioning systems and other sensors, are now being deployed in various areas of operation ranging from recreational, military, farming, package delivery and disaster relief. Other areas of applications for the \ac{CPS} are traffic control, HVAC (heating, ventilation and air conditioning), water management systems and smart medical devices.

These \ac{CPS} are different from typical computer systems because they operate in a physical environment and their properties must conform to laws of physics. Also, \ac{CPS} are built to solve a specific problem and are not multi-purpose like typical computer systems. Other features which set \ac{CPS} apart from traditional computer systems are:

\begin{itemize}
\item  \textbf{\textit{Difficult to update}}: Manufacturers of \ac{CPS} cannot easily update or replace the hardware or the software present in the \ac{CPS}. An extreme example of this use case occurs in a commercial aircraft, which is a safety-critical system with extensive and expensive certification requirements. Software patches or updation of new features on systems like these may take a heavy toll in their day-to-day working and therefore are conducted with caution. Another example is of a smart medical device such as pacemaker, which is used for maintaining the heartbeat of a patient. Updating the software on devices such as these, require the patient to physically visit the healthcare provider as it cannot be done via the Internet. Therefore, it might not always be possible to patch a security vulnerability through a software update.

\item \textbf{\textit{Real-time constraints}}: \ac{CPS} interact with the environment continuously while they are operating. They need to perform actions based on the input gathered from the sensors in real-time. For instance, autonomous cars, based on their surroundings and the path involved in reaching the destination, make certain decisions such as turning the wheels, applying brakes or accelerating. Pacemakers maintain the heartbeat rate by stimulating electrical pulse at the rate of milliseconds. Also, surgical robots have to operate with high operational accuracy as they need to control the timings and area of operation on a patient. All the \ac{CPS} described above have real-time constraints, therefore any hindrance in their continuous operability or change in their correctness and timing behavior will have irreversible consequences.

\item \textbf{\textit{Zero-day attacks}}: With the deployment of modern \ac{CPS} around the world, the security vulnerabilities inherent in them are not fully known. Therefore, the security systems that will be required by these \ac{CPS} need to be able to detect unknown or zero-day attacks, as compared to having a database of known attacks contained in them.

\item \textbf{\textit{Resource constraints}}: \ac{CPS} are built keeping in mind a particular operation, therefore, they are often lacking in computing aspects such as memory, computational power, battery and CPU. Therefore, the security solution that is devised for these devices, need to respect these constraints and operate within the limits. However, even with the resource constraints the behavior modeling capability of the \ac{IDS} should not be sacrificed as it will adversely effect the \ac{CPS} for which the security solution is being devised.

\item \textbf{\textit{Large-scale deployments}}: \ac{CPS} systems such as smart meters are deployed in large scale. Therefore, even a small amount of \acf{FP} - when the \ac{IDS} marks a benign execution of the \ac{CPS} to be malicious - will lead to large amount of manual effort in examining the falsely reported attacks. Also, if an examination requires shutting down the \ac{CPS} after an attack is detected, it might not be the best case to use an \ac{IDS} with high \ac{FP}. Therefore, \ac{FP} should be kept as minimum as possible by the \ac{IDS}.

\end{itemize}
To sum up, the security mechanism which will be developed should be mindful of the constraints of the \ac{CPS} mentioned above. Lately, \ac{CPS} have been targets of security attacks due to their safety-critical nature and relative lack of protection. The advent of interconnected \ac{CPS} to the Internet (also known as the Internet of Things) has exacerbated their vulnerability as they obviate the need for attackers to have physical access to the \ac{CPS}. Attacks on \ac{CPS} such as the smart grid~\cite{skopik2012survey, liu2012cyber} in which the attackers manipulate the smart meter readings have been discovered by researchers. Attacks such as taking control of brakes and steering wheel in smart cars~\cite{checkoway2011comprehensive, woo2015practical} have been demonstrated by hackers. Even, \ac{CPS} such as smart medical devices~\cite{leavitt2010researchers, radcliffe2011hacking}have been targeted by hackers due to their relative lack of security mechanism and wireless communication technology. Therefore, there is a compelling need to protect \ac{CPS} from security attacks. 


\section{Threat Model}
\label{sec:threatModel}
In this thesis, we assume the goal of the attacker is to either alter the benign execution of the \ac{CPS} or make the operator of the \ac{CPS} think that the \ac{CPS} is acting maliciously. The reason behind second kind of attack could be that, the attacker wants the operator to think that the \ac{CPS} has become unstable, as a result of which, the operator might follow some mitigation steps to stabilize the system. During this stabilization, the operator might put the \ac{CPS} on a standstill, until further actions that need to be carried out are agreed upon. Therefore, the actions that the operator is performing to stabilize the \ac{CPS}, might be the goal of the attacker. The reason behind both attacks could be either monetary or collecting valuable information from the \ac{CPS} which could then be used in other kind of exploitation. For instance, in an \ac{UAV}, an attacker could spoof the values of distance traveled being sent to the \ac{GCS}. As a result, the operator might issue commands for the \ac{UAV} to descend and land on the ground, until the cause of the deviation is known. In order to achieve the goals, attacker can tamper with either the communication channel or the control logic present in the controller. We now discuss the access and capabilities of the attacker.

{\bf Access}: The term \ac{SUT} in this study is used to represent the system on which the analysis is performed. It is assumed that the attacker has the capability to gain read and write access to the communication channel between the \ac{SUT} and the controller. Using this access, the attacker can modify the contents or add data packets being transferred. This assumption is realistic as previous work~\cite{davanian2017diversity, ericsson2010cyber} has shown that such access is rather easy to get.

Further, it is assumed that the attacker has the access to the control system of \ac{SUT} \cite{alemzadeh2016targeted}, which means that the application code can be modified to suit the attacker's needs. Also, an assumption that the attacker cannot modify the operating system kernel or the device firmware is made. This can be ensured by using code signing or trusted computing hardware if it is available. 

{\bf Capabilities}: It is assumed that the attacker, using access to the communication channel, can perform two types of attacks. The first one is spoofing, where the contents of the data packets can be modified, and the second one is flooding where the number of data packets being sent to the controller can be increased. The attacker can also perform physical attacks on the \ac{CPS}, for example by rebooting it at arbitrary points in time. 

With the access to the control system of \ac{SUT}, it is assumed that the attacker can change the control logic to introduce the attack in the \ac{CPS} to accomplish the goal of altering the benign execution of a \ac{CPS}. However, as an attacker is likely to want to remain stealthy, it is more likely to make small changes to the program rather than large-scale changes such as replacing the entire program with their own. 
 
For this research, attacks which compromise the confidentiality/privacy of the \ac{CPS} such as network attacks - \ac{DoS} or message dropping attacks - are \textit{not} considered, because these attacks can be detected by network security mechanism. Also, only attacks that change the correlation between the logical properties are considered. Therefore, attacks which do not create an impact on the correlation between logical properties are not considered. 

\section{Approach}
\label{sec:Approach}
Now, we discuss the approach that we take towards securing the \ac{CPS}. \ac{IDS} are being used for protecting computer systems from security attacks, including CPS~\cite{lu2015towards, mitchell2015behavior, bernieri2016testbed}. \ac{IDS} work by monitoring the activity of the system for which they are deployed and raise alarm when they detect a malicious intent. Traditional forms of \ac{IDS} are signature-based, where signatures of known attacks are compared against the operations of the system to identify attacks. Unfortunately, signature-based \ac{IDS} are a poor fit for \ac{CPS} as the attacks are often tailored to each kind of \ac{CPS}, and hence cannot be described by generic signatures. Further, due to the remote and often disconnected nature of their operation, the attack database in \ac{CPS} cannot be updated frequently unlike traditional computer systems. Finally, a motivated attacker can launch hitherto unknown attacks against a \ac{CPS}, thereby evading detection by signature-based schemes. 

In contrast to signature-based \ac{IDS}, anomaly-based \ac{IDS} extract a model of a system's behavior and detect any deviations from the extracted model as an attack. Such \ac{IDS} do not need an attack database, and can hence detect hitherto unknown attacks. Because \ac{CPS} have constrained behaviors, it is often straightforward to derive anomaly-based \ac{IDS} for them, making these systems a good match for \ac{CPS}. Unfortunately, anomaly-based systems exhibit high rates of false-positives in practice, as learning a stable model of the system is often challenging. Therefore, some researchers have proposed using physics-based models for anomaly detection models for intrusion detection in \ac{CPS}~\cite{mitchell2012specification,mitchell2014adaptive,choudhari2013stability,chen2018learning,zohrevand2016hidden}. The notion is that because \ac{CPS} interact closely with their physical environments, they need to follow laws of physics, which can in turn be used as the detection model. 
Efforts have been made to use the physical properties of the power grid~\cite{choudhari2013stability,paul2014unified}, \ac{UAV}~\cite{mitchell2012specification} and water treatment systems~\cite{adepu2016using} to build a model which represents the expected behavior of the \ac{CPS}. However, in prior work~\cite{mitchell2012specification,mitchell2014adaptive,choudhari2013stability,chen2018learning,zohrevand2016hidden, adepu2016using, paul2014unified}, the \ac{IDS} is designed specifically for a particular \ac{CPS}. Therefore, the above solutions cannot be easily generalized to other \ac{CPS}, as the process of finding an appropriate model is both time consuming and effort intensive for developers. 

In this work, the {\em logical correlations} among the physical properties of the \ac{CPS} are considered as the model for anomaly-based \ac{IDS}. The hypothesis is that physical properties exhibit deterministic and predictable correlations among themselves, as they have to adhere to the laws of physics. For example, consider the case of an \ac{UAV}, which needs to follow Newton's laws of motion during flight. Some physical properties of an \ac{UAV} are: distance traveled, altitude, speed, and flight time. When an \ac{UAV} is flying at a fixed altitude, it has a non-zero speed due to which the distance traveled and flight time increases, while the battery life left in the \ac{UAV} decreases. These relationships encompass the logical correlations among the physical properties of the \ac{UAV}. If during flight it is observed that the battery life left in the \ac{UAV} is not decreasing while the speed of the \ac{UAV} is non-zero, this would imply that there is some anomaly in the system, which potentially indicates an attack.

\section{Hidden Markov Models}
\label{sec:HMM_explain}

In this study, an anomaly-based \ac{IDS} is built which internally uses an \ac{HMM} to find logical correlations among the physical variables in a system. \ac{HMM} are useful for systems which can be represented by sequences or time series. An \ac{HMM} is a finite model that can be used to describe a probability distribution over an infinite number of possible sequences in a given system~\cite{eddy1996hidden}.

Unlike a simple Markov model, an \ac{HMM} is composed of a number of hidden states. Each hidden state 'emits' symbols according to emission probabilities, and the states are interconnected by state-transition probabilities. Starting from an initial state, a sequence of states is generated by moving from state to state according to the state-transition probabilities until an end state is reached. Each state then emits symbols according to that state's emission probability distribution, creating an observable sequence of symbols.
More formally, an \ac{HMM} can be represented by $\pi$, A, $\theta$ where $\pi$ represents the starting probability of the transitions between the hidden states, while the transition probability matrix is denoted by A and $\theta$ represents the emission probability of the hidden states.

\ac{HMM} are a good fit for problems in which i) the model parameters and observed data are present, and there is a need to estimate the sequence of hidden states; ii) the observed data is given and the model parameters are to be estimated, and iii) the information of model parameters and observed data is present while there is a need to find the likelihood of data. Therefore, in this study, we intend to use \ac{HMM} for the third kind of problem, i.e., determining the likelihood of current observed data belonging to the predefined model's parameters. In order to do so, the values of correlated physical properties of the system can be fed into an \ac{HMM}, which can then infer the correlations between them. These correlations can be used to determine the likelihood of the current observed data as stemming from the model and its parameters. Any deviation could be signaled as an anomaly and a possible security attack.

\ac{HMM} act as the core of intrusion detection module mainly because they are capable of finding data patterns in high dimensional, non-linear time series based systems. Also, \ac{HMM} work by creating hidden states and then transitioning between them which is very similar to the operations of \ac{CPS} system, which are typically modeled as state machines.
Unlike techniques such as correlation coefficients, \ac{HMM} are also highly resilient to noise and outliers. 
For instance, Krotofil et al.~\cite{krotofil2015process} use \acf{PCC} to determine correlation for the cluster entropy. \ac{PCC} measures linear correlation among the variables, therefore is not suitable for multidimensional non-linear data. Also the variables undergoing \ac{PCC} must be either based on interval or ratio scale, making this approach much less generic. Chen at el.  \cite{chen2018learning} employ \ac{SVM} to detect an anomaly in a time series based system. Unfortunately, \ac{SVM} do not work well with time series data, because they work with a snapshot of the state and classify it into a class. However, by manipulating the input feature vector to the \ac{SVM} in such a way that it encapsulates the time factor, authors use it for anomaly detection. On the other hand, Aliabadi et al. ~\cite{aliabadi2017artinali} use Frequent Item Set Mining algorithm which does not model the system, but mines the data under different events. Unfortunately, they do not consider the physical properties, except time, of the \ac{CPS}. Iturbe et al. \cite{iturbe2017feasibility} use \ac{PCA} which infers correlation among the variables and is better suited for linear correlations, as it works by generating orthogonal projections. However, for non-linearly correlated data as in our case, \ac{PCA} is not able to find correlations.


\section{Contributions}
\label{sec:Contributions}

This research proposes a generic intrusion detection system capable of detecting security attacks by inferring the logical correlations of the system and checking if they adhere to a predefined framework. \ac{HMM}  are used to automatically infer the logical correlations among the physical properties of the system with no a priori knowledge of the physical laws adhered to by the system or any intervention by the programmer. The \ac{HMM} identifies a state as malicious by detecting either an undesired data correlation or lack of an expected data correlation among its physical properties. \ac{HMM} are used as they are good at detecting outliers, and are typically used to model time-based systems (~\autoref{sec:HMM_explain}).  
Though other papers have used logical correlations to detect anomalies~\cite{iturbe2017feasibility,krotofil2015process,chen2018learning,zohrevand2016hidden}, none of them have used \ac{HMM} as the core to build a generic \ac{IDS}. \textit{To the best of our knowledge, \ac{CORGIDS} is the first generic intrusion detection system which uses \ac{HMM} to infer logical correlations exhibited by the system to determine if an intrusion has occurred.} Our contributions are summarized by the following set of actions and outcomes:

\begin{enumerate}

\item Proposed the use of logical correlations exhibited by the physical properties of a \ac{CPS} to detect intrusions, and the use of \acf{HMM} to infer the logical correlations.

\item Designed a \acf{CORGIDS} prototype using \acf{HMM} to detect intrusions. Also, demonstrated its use on two behaviorally different \ac{CPS} test-beds, namely i) an \ac{UAV}, and ii) a \acf{SAP}.

\item Evaluated the effectiveness of \ac{CORGIDS} by performing five targeted attacks (\autoref{ch:Attacks}) and three arbitrary attacks (\autoref{ch:comparisonwithrelatedwork}) on the above mentioned \ac{CPS}. Found that \ac{CORGIDS} is successfully able to detect all five attacks, and has lower \acf{FP} and \acf{FN} rate than other intrusion detection techniques (described in \autoref{sec6:Evaluation}).

\item Performed a comprehensive comparison with a technique in the related work which design a generic IDS to detect intrusion. Results exhibits that \ac{CORGIDS} achieves significantly lower \ac{FP} and \ac{FN} for both targeted and arbitrary attacks (expect artificial delay insertion attack) when compared to this related work. This comparison and its results are described in ~\autoref{ch:comparisonwithrelatedwork}.


\end{enumerate}

\section {Publications}
The work in this thesis has been published in the following research paper:
\begin{itemize}
\item "CORGIDS: A CORRELATION-BASED GENERIC INTRUSION DETECTION SYSTEM" Ekta Aggarwal, Mehdi Karimibuiki, Karthik Pattabiraman and Andr\'e Ivanov, Proceedings of the 2018 Workshop on Cyber-Physical Systems Security and Privacy 2018 (Acceptance Rate: 45\%)
\end{itemize}

The remainder of this thesis is structured as follows. Chapter 2 explores related work, and Chapter 3 explains the approach used to build an \ac{IDS}, \ac{CORGIDS}, which is used  to detect intrusion. Then, Chapter 4 presents the test-beds used for evaluation of \ac{CORGIDS}. Next, Chapter 5 describes the attacks which are emulated on the experimental test-beds and Chapter 6 reports the results from the attacks seeded in Chapter 5. Chapter 7 describes a quantitative comparison of \ac{CORGIDS} with its related work. Chapter 8  discusses the limitations and applicability of \ac{CORGIDS} and finally Chapter 9 concludes the thesis by describing conclusion and future work.
%%%%%%%%%%%%%%%%%%%%%%%%%%%%%%%%%%%%%%%%%%%%%%%%%%%%%%%%%%%%%%%%%%%%%%
%%%%%%%%%%%%%%%%%%%%%%%%%%%%%%%%%%%%%%%%%%%%%%%%%%%%%%%%%%%%%%%%%%%%%%
%%%%%%%%%%%%%%%%%%%%%%%%%%%%%%%%%%%%%%%%%%%%%%%%%%%%%%%%%%%%%%%%%%%%%%

\endinput
