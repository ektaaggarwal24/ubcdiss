%% The following is a directive for TeXShop to indicate the main file
%%!TEX root = diss.tex

\chapter{Abstract}

\ac{CPS} consist of software and physical components which collaborate and interact with each other continuously. \ac{CPS} deployed in security-critical scenarios such as medical devices, autonomous cars and smart homes have been targets of security attacks due to their safety-critical nature and relative lack of protection. Anomaly based \ac{IDS} using data, temporal, and logical correlations have been proposed in the past. But none of the approaches except the ones using logical correlations take into account the main ingredient in the operation of \ac{CPS}, namely the use of physical properties. On the other hand, \ac{IDS} that use physical properties either require the developer to define invariants manually, or have designed their \ac{IDS} for a specific \ac{CPS}. This study proposes a \ac{CORGIDS}, a generic \ac{IDS} capable of detecting security attacks by inferring the logical correlations of the physical properties of a \ac{CPS}, and checking if they adhere to the predefined framework. A \ac{CORGIDS}-based prototype is built and used for detecting attacks on two example \ac{CPS}s - \ac{UAV} and \ac{SAP}. It is found that \ac{CORGIDS} achieves a precision of  95.70\%, and a recall of 87.90\%, while detecting attacks with modest memory and performance overheads.

% Consider placing version information if you circulate multiple drafts
%\vfill
%\begin{center}
%\begin{sf}
%\fbox{Revision: \today}
%\end{sf}
%\end{center}
