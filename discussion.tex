%% The following is a directive for TeXShop to indicate the main file
%%!TEX root = main.tex

% ===========================================================================================
\chapter{\textbf{Discussion}}
\label{sec7:Discussion}
This chapter first details the threats to validity followed by the generic applicability of \ac{CORGIDS} and how it can be circumvented.

\section{Threats to validity}
Threats to validity consist of proofs which can be used to support a claim which is contrary to the context supported by a particular study. Threats to validity help in bringing out the potential areas/scope within the research which might pose a threat to it.

The three threats to validity that are considered in this research are:

\begin{itemize}
\item An \textit{Internal threat} to this work is the consideration of only five targeted attacks to gauge its performance. For instance, this study does not experiment with other types of targeted attacks such as dropping attacks or \ac{DoS} attack. This threat is attempted to be mitigated by choosing attacks which are very different in nature and exploit different domains of the test-beds. Also, arbitrary attacks are used which form the building blocks of the attacks that might occur in future. Another internal threat is the use of simulations to gauge the effectiveness and performance of \ac{CORGIDS}. Though this threat is substantial, it is attempted to be mitigated by keeping the simulations as unbiased as possible. For instance, for each flight of an \ac{UAV}, the number of way-points, latitude and longitude of each way-point and altitude were randomized. However, for future work, \ac{CORGIDS} will also be evaluated on a real test-bed. 

\item An \textit{External threat} consists of the use of only two test-beds from the \ac{CPS} domain to prove that this approach is effective and general. However, finding test-beds which are publicly available (open-source) and also are security critical is a difficult task. This threat was attempted to be mitigated by choosing two test-beds which were entirely different in behavior and utility. An \ac{UAV} is used for flight operations and uses physical laws of motion for operation, while the SAP is a medical device and uses biological properties of the human body to calculate the appropriate amount of insulin to be injected. 

\item Finally, the \textit{Construct  threat} to validity is the use of only \ac{FP}, \ac{FN}, precision and recall for the evaluation of \ac{CORGIDS}. However, these metrics are also used substantially by prior work in this area and therefore are valid for comparison purposes. 
\end{itemize}

\section{A Generic IDS}
Building a generic \ac{IDS} for systems exhibiting correlation is one of the key contributions of this thesis. This approach utilizes the correlation exhibited by logical properties, for instance speed, distance traveled, altitude, battery in the \ac{UAV} and flight time are some logical properties. These values are dependent on each other and change according to only a predefined framework, for example, the laws of physics for an \ac{UAV}. However, \ac{CORGIDS} cannot be applied to those systems which do not exhibit such correlations. For instance, systems except \ac{CPS} and financial systems, in which no correlation can be found between its variables/properties are not the candidates for using \ac{CORGIDS}.

\section{Circumventing \ac{CORGIDS}}
As discussed in Section~\ref{sec:threatModel}, it is assumed that the attacker has capabilities which can be used to plant an attack on the \ac{SUT}. An attacker who knows about the internal working of the system can circumvent the intrusion detection done by \ac{CORGIDS}. An example scenario in which the attack will be undetected is, when the attacker hacks the \ac{SUT} and changes the logging module of \ac{CORGIDS} to send the correct correlated values of physical properties irrespective of them being faulty at that point of time. If the attacker were to continue this operation throughout the \ac{UAV} flight, \ac{CORGIDS} would not be able to detect intrusion, because correctly correlated values will be received by \ac{GCS}. However, updating all the correlated values at every second during the entire flight is constrained by power consumption, time and effort \cite{krotofil2015process}. So, the case would likely be that the attacker would  not be able to forge the values throughout entire duration, thus leading to some discrepancy in values of logical properties which would be flagged by \ac{CORGIDS}. Another use case where the IDS could fail is when the attacker manipulates the \ac{CPS} but keeps it very close to the benign behavior. As \ac{CORGIDS} will not have any correlation which is deviating by a large amount from its original state, the attack will most probably go unnoticed. However, if the attacker attempts to perform this type of attack, it will lead to very little deviation from the original benign behavior, thus leading no harm to the \ac{CPS}.

Another point to note is that the effect of varying a single correlated property for intrusion detection was demonstrated in this study. Varying multiple properties in the system will have a similar effect and will lead to an unbalanced correlation which will be spotted by the \ac{HMM}. Also, by varying the rate of increase or decrease of the anomalous correlated property, variations in the log probability of the current system state will arise which will be marked malicious. However, if log probability of the current malicious state is very close to the benign state's log probability, it is likely that \ac{CORGIDS} would not be able to distinguish between these two states, and thus the attack would not be detected.
\endinput
=====================================================================
% EOF


