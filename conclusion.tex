%% The following is a directive for TeXShop to indicate the main file
%%!TEX root = main.tex

% ===========================================================================================
\chapter{\textbf{Conclusion and Future work}}
\label{sec8:Conclusion}

\section{Conclusion}
\acf{CPS} systems exhibit correlation in their logical properties as they need to interact with the physical environment, which is subject to the laws of physics. These correlations can be monitored to label out anomalous activities in the system. Lately, attackers have targeted \ac{CPS} owing to their loose security control measures. Though, the use of physical properties (logical properties) of the \ac{CPS} to detect an intrusion has gained prominence lately, all the solutions discussed in this study either use manually defined physical rules/invariants, or dynamically build the invariants but only for a specific \ac{CPS}. To fill this gap, this study proposes a generic \ac{IDS}, \ac{CORGIDS}, designed for systems which exhibit correlations, that uses \acf{HMM} for extracting the correlations. \ac{HMM} are much more resilient to outliers and noise compared to other techniques, and do not presuppose a distribution of the properties, making them generic.
Use of \ac{CORGIDS} was demonstrated on two diverse \ac{CPS} test-beds, an \ac{UAV} and a \ac{SAP}. It was found that \ac{CORGIDS} is able to detect intrusion with significantly less \acf{FP} and \acf{FN} and with more precision and recall when compared with other \ac{IDS}. For an apples-to-apples comparison, \ac{CORGIDS} was compared with ARTINALI, a generic \ac{IDS} using Data-Event-Time interplay to detect intrusion. Both targeted and arbitrary attacks from ARTINALI and \ac{CORGIDS} were used to provide an even playing ground for both the techniques. Results from this comparison exhibited that \ac{CORGIDS} performed with lower \ac{FP} and \ac{FN} for all the targeted and arbitrary attacks (except the artificial delay insertion attack) as opposed to ARTINALI.

\section{Future work}
As part of the future work, CORGIDS could be implemented on a real test-bed. This will help in understanding the differences between real world testing and simulations, if any. There may be a case where the real world testing leads to more noise in the system traces due to environmental factors. Therefore, when using the real test-bed traces for training purposes, they should either be refined to remove noise or can be directly used, as they will provide a basis for the IDS to learn/observe the environment as it is. For instance, in this study Ardupilot's \ac{SITL} simulator is used for experiments, therefore in future the same experiments could be modified for the real UAV platform. Else, a completely new test-bed, for example, a real rover could be used to gauge \ac{CORGIDS} efficacy. As \ac{CORGIDS} is a generic system and requires that the \ac{SUT} contains the correlation among properties, it can be applied to a rover.

Another possible extension of this work could be design of a mitigation technique. This study only concentrates on detection of attacks, that is, once the attack happens \ac{CORGIDS} just alerts the controller about it. The rescue or the mitigation is left on to the operator who will most probably manually try to control the \ac{CPS} under attack. Mitigation technique could come in handy at this time, where the mitigation software which is coupled along with \ac{CORGIDS} knows what steps to take once the \ac{CPS} is acting maliciously. This technique would offer immediate resolution of conflict, rather than waiting for the operator to take an action.

 \endinput
%=====================================================================
% EOF