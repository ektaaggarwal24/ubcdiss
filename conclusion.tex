%% The following is a directive for TeXShop to indicate the main file
%%!TEX root = main.tex

% ===========================================================================================
\chapter{\textbf{Future Work and Conclusion}}
\label{sec8:Conclusion}

CPS systems exhibit correlation between their logical properties as they need to interact with the physical environment, which are subject to the laws of physics.

As future work, CORGIDS could be extended to include an apple-to-apple comparison with other related techniques such as ARTINALI, Zohrevand et.al. and Chen et. al. which dynamically extract invariants. This comparison could consist of the above three techniques detect intrusion on same set of attacks while being adapted for a certain CPS.

Lately, attackers have targeted CPS owing to their loose security control measures. Though, the use of physical properties (logical properties) of the CPS to detect an intrusion has gained prominence lately, all these solutions either use manually defined physical rules, or dynamically build the invariants but only for a specific CPS. This study proposed a generic IDS, CORGIDS, designed for systems which exhibit correlations, that uses Hidden Markov Models (HMMs) for extracting the correlations. HMMs are much more resilient to outliers and noise compared to other techniques, and do not presuppose a distribution of the properties, making them generic.
The use of CORGIDS wasd demonstrated on two diverse CPS. It was found that CORGIDS is able to detect intrusion with significantly less FPs and FNs and with more precision and recall when compared with other IDS. 
 \endinput
%=====================================================================
% EOF