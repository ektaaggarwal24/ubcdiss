%% The following is a directive for TeXShop to indicate the main file
%%!TEX root = main.tex

% ===========================================================================================
\chapter{\textbf{Conclusion and Future work}}
\label{sec8:Conclusion}

\section{Conclusion}

\ac{CPS} have gained ubiquitous popularity over the masses in the past few years. They are different from traditional computer systems as they need to interact with the physical environment, which is subject to the laws of physics. Owing to the constraints that these \ac{CPS} possess, they have become the targets of attackers who exploit them due to the insufficient protection and interconnectedness. \ac{IDS} which are specifically tailored for \ac{CPS} have been devised in the past to secure these systems from tailored attacks. The key point to note here is that the physical properties of the \ac{CPS} depict the current state of the system. These properties can be used to study the correlation exhibited and further can be used to detect presence of an abnormal activity. Using this key insight, we build \acf{CORGIDS}, a generic \ac{IDS} which dynamically generated the physical invariants using the physical properties of the \ac{CPS}. Also, we test the prototype on two behaviorally different \ac{CPS} test-beds, namely i) an \ac{UAV}, and ii) a \acf{SAP}. We find that \ac{CORGIDS} produces significantly lower \ac{FP} and \ac{FN} for the 5 targeted attacks that were tested on the test-beds.

Though, the use of physical properties of the \ac{CPS} to detect an intrusion is used to build \ac{IDS} previously, all the solutions discussed in this study either use manually defined physical rules/invariants, or dynamically build the invariants but only for a specific \ac{CPS}. Both of the invariant generation techniques mentioned require more developer effort and time, and manually defining invariants specifically needs an in-depth knowledge of the \ac{CPS} or else the invariants defined for intrusion detection could themselves be faulty. To fill this gap, this study proposes a generic \ac{IDS}, designed for systems which exhibit correlations, like \ac{CPS}. Therefore, though an \ac{UAV} and \ac{SAP} are completely different in working, behaviorally and uses, \ac{CORGIDS} can still be applied as it tries to infer the correlation present within the set of physical properties for each of the test-bed.
 
The related work which dynamically generates physical invariants for \ac{CPS} uses algorithms/models such as i) \textit{\acf{PCC}}, which is useful for measuring linear correlation among data variables and not particularly made for non-linear multi-dimensional data, ii) \textit{\acf{SVM}} were used to build a invariant which distinguished between malicious and benign behavior. However, these are models are not well suited for time-series based systems, iii) \textit{Data-Time-Event interplay}, was used to deduce how the values of various data points in different events were changing over the curse of functioning of the \ac{CPS}, but it didn't learn to deduce how the \ac{CPS} was behaving. We however on the other hand, used \acf{HMM} to identify the correlations between non-linear multi-dimensional data of the \ac{CPS}. \ac{HMM} are much more resilient to outliers and noise compared to other techniques, and do not presuppose a distribution of the properties, making them generic.

For a comprehensive comparison, \ac{CORGIDS} was also compared with ARTINALI, a generic \ac{IDS} using Data-Event-Time interplay to detect intrusion. Attacks from both the \ac{IDS} were combined to generate a super set of attacks on which ARTINALI and \ac{CORGIDS} were tested. Particularly, both targeted and arbitrary attacks from ARTINALI and \ac{CORGIDS} were used to provide an even playing ground for both the techniques. Results from this comparison exhibited that \ac{CORGIDS} performed with lower \ac{FP} and \ac{FN} for all the targeted and arbitrary attacks (except the artificial delay insertion attack) as opposed to ARTINALI. Our results showed 
that as \ac{CORGIDS} during training phase deduces the behavior of the \ac{CPS} under test, and ARTINALI on the other hand focuses on gathering the Data-Event-Time interplay, which does not include the correlation among different physical properties, \ac{CORGIDS} was able to detect more attacks as the attacks on \ac{CPS} ultimately led to change in physical properties.


\section{Future work}
Below mentioned are a few directions which this work could take in future.

\subsection{Implementation of \ac{CORGIDS} on a real test-bed}
As part of the future work, \ac{CORGIDS} could be implemented on a real test-bed. This will help in understanding the differences between real world testing and simulations, if any. There may be a case where the real world testing leads to noise in the system traces due to environmental factors. Therefore, when using the real test-bed traces for training purposes, they could be either refined to remove noise or could be directly used, as they will provide a basis for the IDS to learn/observe the environment as it is. For instance, in this study Ardupilot's \ac{SITL} simulator is used for experiments, therefore in future the same experiments could be performed for the real \ac{UAV} platform. The experiment on a real \ac{UAV} platform will allow a detailed comparison in the differences in performance of \ac{CORGIDS} - \ac{FP}, \ac{FN}, performance and memory overheads - when compared to its simulated version. Else, a completely new test-bed, for example, a real rover could be used to gauge \ac{CORGIDS} efficacy. As \ac{CORGIDS} is a generic system and requires that the \ac{SUT} contain the correlation among properties, it can be easily modified to be used for intrusion detection in rovers.

\subsection{Testing efficacy of \ac{CORGIDS} on other attacks}
Another area of improvement for this study could be the addition of experimentation of additional attacks. As described in ~\autoref{ch:Attacks}, the work done up-til now considers only 5 targeted attacks on the 2 test-beds. This work could be expanded by testing \ac{CORGIDS} on other network attacks, \ac{DoS} and message dropping attacks. Also, attacks that are used by related work in their study, can be used to gauge efficacy of \ac{CORGIDS}.

\subsection{Identifying the malicious property using \ac{CORGIDS}}
An interesting addition to \ac{CORGIDS} can be the ability of detecting the physical property which is destabilizing the \ac{CPS}. Currently, \ac{CORGIDS} by monitoring the correlation of the physical properties of the \ac{CPS}, is able to detect if there is a malicious activity in the \ac{CPS}. However, it does not identify the physical property which is breaking the correlation, for example, in distance spoofing attack in \ac{UAV}, faulty values of \textit{distance traveled} break the correlations and in \ac{SAP}, the insulin tampering attack results in faulty \textit{insulin} values to be injected in the patient. Therefore, as an extension to the current work, if we could pinpoint the faulty physical property using the abnormal correlations, necessary action could be taken by the operator to investigate the current situation. This diagnosis from \ac{CORGIDS} will help in decreasing the operator's time and manual effort involved when an attack is detected.

\subsection{Coupling of an automated mitigation technique with \ac{CORGIDS}}
Another possible extension of this work could be integration of a mitigation technique along with \ac{CORGIDS}. As \ac{CORGIDS} is responsible for only  detection of attacks in \ac{CPS}, that is, once the attack happens \ac{CORGIDS} alerts the controller about it. As of now, the rescue or the mitigation is left onto the operator who will most probably manually try to control the \ac{CPS} under attack. However, an integrated mitigation technique could come in handy at this time, where the mitigation software which is coupled with \ac{CORGIDS} is intelligent enough to know what steps are required to be executed in order to  stabilize the \ac{CPS} which was acting maliciously. This technique could offer immediate resolution of conflict, rather than waiting for the operator to take an action which will require time. The automated mitigation process could be also based upon the physical properties of the \ac{CPS} and always monitors the current physical status and comes into play once \ac{CORGIDS} raises an alarm. For instance, consider an attack where the attacker modifies the controller in such a way that it is sending faulty acceleration values. Particularly, the new malicious acceleration values being sent to the actuators are causing an \ac{UAV} to descend at an abnormal speed which could eventually lead to a crash. Due to change in acceleration (physical property) of an \ac{UAV}, \ac{CORGIDS} will detect an intrusion and raise an alarm. At this time the automated mitigation technique which was monitoring all the changes in the physical environment kicks in and tries to stabilize the system by either activating the fail-safe mode, if any, else putting the \ac{UAV} in a hover state which breaks its steep descend and gives the operator a stable system which can then be safely navigated to the desired position.

 \endinput
%=====================================================================
% EOF